
\chapter{INTRODUÇÃO}

A busca pelo conhecimento de diferentes materiais sempre foi uma necessidade da civilização humana. A própria história recebeu marcos após a descoberta e manipulação de metais, e suas ligas. A nossa sociedade se beneficiou com os metais, sejam eles puros como o ouro, cobre, ferro, como também com suas ligas latão, bronze, e aços. Através desses materiais, foi possível a confecção de pequenos utensílios para decoração, moedas, instrumentos cirúrgicos, ferramentas úteis para experimentos científicos, até grandes monumentos, pontes, edifícios, foguetes espaciais, etc. 

Muito se foi explorado e ainda há de ser em ligas convencionais, isto é, ligas com um único elemento representando a matriz e a adição de outros elementos com objetivo de melhorar as propriedades da liga. Um exemplo bastante comum é o que temos hoje como os aços inoxidáveis, onde sua matriz é basicamente ferro, mas com pequenas adições de cromo e níquel há um aumento considerável em sua resistência a corrosão. 

Além da resistência a corrosão, diversos outras melhorias nas propriedades dos metais são desejadas, como elevada resistência mecânica, resistência a abrasão, tenacidade, resistência em altas temperaturas, resistência a corrosão sob tensão, baixa densidade, etc \cite{jien2006recent}.               
Outro tipo de pesquisa que vem sendo realizado, com o mesmo objetivo de melhoria de propriedades listados acima, é o conceito de ligas metálicas baseadas na combinação de múltiplos elementos. Essas ligas são conhecidas como "Ligas de Alta Entropia", traduzido do termo em inglês "High Entropy Alloys - HEAs" \cite{yeh2004nanostructured}. Esse novo conceito de ligas faz com que o estudo de dos diagramas de fase seja deslocado de suas extremidades para o seu centro, uma vez que a a composição química de sua matriz é equiatômica ou ao menos tende a ser próximo disso. Isto significa um aumento exponencial de possibilidades de ligas a serem exploradas.



\bigskip
