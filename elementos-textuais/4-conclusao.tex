
\chapter{CONCLUSÃO}

O desenvolvimento de novas ligas de Alta Entropia podem oferecer diversas opções com propriedades tão boas quanto, ou até melhores, que ligas convencionais comuns. Como essas ligas são ainda um tema considerado recente na linha de pesquisa e desenvolvimento, as informações sobre essas ligas são limitadas. Sendo assim, para uma possível redução de custos na sua fase inicial de desenvolvimento, uma excelente alternativa é o uso de inteligência artificial utilizando os conjuntos de dados disponíveis na literatura, para obter uma previsão de que tipo de microestrutura será esperada para uma faixa de composição química arbitrária. 

No presente estudo, foi verificado que com uma limitada quantidade de amostras de 382 dados, podemos obter uma acurácia de um pouco mais de 80\% utilizado os dados rebalanceados com a sobreamostragem, e uma acurácia de quase 60\% utilizando 30\% dos dados separados para teste. Um modelo de aprendizado de máquina pode ser aplicado em diversas categorias, seja para área da saúde, legislação, mercado financeiro, entre outros. No caso deste trabalho, o uso de inteligência artificial, contribuiu para prever a microestrutura de ligas de alta entropia, utilizando a sua composição química em porcentagem, os valores de entropia de mistura, constante de rede, temperatura de fusão, entre outras características citadas nos materiais e métodos.


Em projetos de aprendizado de máquina mais evoluídos, o conjunto de dados costuma estar na casa de milhares. E no caso deste estudo com aproximadamente 400 dados, podemos obter uma classificação satisfatória de ligas de alta entropia. Assim, para os estudos futuros, conforme  maior disponibilidade dos dados, o uso de inteligência artificial poderá ser um forte aliado para redução de custos em pesquisas e desenvolvimento de novas ligas. O compartilhamento e a cooperatividade entre diferentes centros de pesquisas é o fator fundamental para evoluir a linha de pesquisa acadêmica em sinergia com uso de ferramentas de aprendizado de máquina, pois o uso desses algoritmos permitem que a comunidade acadêmica desenvolva novas ligas com mais assertividade a partir de simulações, e caso necessário corrigir o modelo conforme necessário. 



% Onde se expõe o fechamento das ideias do estudo, são apresentados os resultados da pesquisa, e partindo da análise destes resultados, tiram-se as conclusões e se for necessário, as sugestões relativas ao estudo. \\

% Observação: É opcional a apresentação dos desdobramentos relativos à importância, síntese, projeção, repercussão, encaminhamento e outros.